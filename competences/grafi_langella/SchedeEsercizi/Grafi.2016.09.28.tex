\documentclass[12pt]{article}
\usepackage[T1]{fontenc}
\usepackage[latin1]{inputenc}
\usepackage[italian]{babel}
\usepackage{latexsym}
\usepackage{epsfig}
\usepackage{fancybox}
\usepackage{rotating}
\usepackage{graphicx}

\textwidth 15.5cm
\textheight 22.5cm
\topmargin 0cm
\evensidemargin 0in
\oddsidemargin 0in


\def\bbbr{{\rm I\!R}}
\def\bbbm{{\rm I\!M}}
\def\bbbn{{\rm I\!N}}

\def\sm{\setminus}
\newcommand {\ol}[1]{\overline{#1}}



\def\La{{\cal \Leftarrow}}
\def\Ra{{\cal \Rightarrow}}


\newtheorem{Teo}{Theorem}[section]
\newtheorem{Con}[Teo]{Conjecture}
\newtheorem{Que}{Question}
\newtheorem{Cor}[Teo]{Corollary}
\newtheorem{Property}[Teo]{Property}
\newtheorem{Pro}[Teo]{Proposition}
\newtheorem{Cla}[Teo]{Claim}
\newtheorem{Lem}[Teo]{Lemma}
\newtheorem{teo}{Teorema}
\newtheorem{Obs}[Teo]{Observation}
\newtheorem{Nota}{Nota dello studente}

\pagestyle{empty}
 
\begin{document}
\title{\textbf{Scheda esercizi: Grafi\\ Esame del 28/09/2016}}
\author{}
\date{}
\maketitle

{\sc \bf Problema~6\/\ \ (15 punti)\/:}

Si consideri il grafo, con pesi sugli archi,
riportato in figura.

\begin{figure}[htb]   % h=here; t=top; b=bottom;
\begin{center}
  \leavevmode
  \psfig{figure=input.pdf, height=7.0 true cm}
\end{center}
\end{figure}

\begin{itemize}
   \item[6.1.{\bf (1pt)}] Dire, certificandolo, se il grafo
                          \`e planare oppure no.
                          In ogni caso, disegnare il grafo
                          in modo da minimizzare il numero
                          di incroci tra archi.
   \item[6.2.{\bf (1pt)}] Dire, certificandolo, se il grafo
                          ottenuto da $G$ sostituendo l'arco $hx$ con un arco $qx$
                          \`e planare oppure no.
   \item[6.3.{\bf (1+1pt)}] Trovare l'albero dei cammini minimi
                          dal nodo $s$. Esprimere la famiglia di tali alberi.
   \item[6.4.{\bf (1pt)}] Trovare un albero ricoprente
                          di peso minimo.
   \item[6.5.{\bf (4pt)}] Per ciascuno dei seguenti archi dire, certificandolo, se esso appartenga a (tutte/ a nessuna/ a qualcuna ma non a tutte) le soluzioni ottime: \emph{dk, ax, es}.
   \item[6.6.{\bf (2pt)}] Trovare tutti gli alberi ricoprenti
                          di peso minimo.
                          (Dire quanti sono e specificare con precisione
                           come generarli).
   \item[6.7.{\bf (1pt)}] Trovare un massimo flusso dal nodo $s$
                          al nodo $t$.
   \item[6.8.{\bf (1pt)}] Certificare l'ottimalit\`a del flusso
                          massimo dal nodo $s$ al nodo $t$.
   \item[6.9.{\bf (1+1pt)}] Fornire (con certificato di ottimalit\`a) il flusso
                          massimo dal nodo $s$ al nodo $q$.
\end{itemize}

\bigskip
{\bf risposte.} 

\textbf{(1pt)} Il fatto che $G$ sia planare
pu� essere messo in evidenza
esibendo il planar embedding in figura.

\begin{center}
\mbox{
  \psfig{figure=planarity.pdf, height=7.0 true cm}
}
\end{center}

Per la ricerca di alberi ricoprenti di peso minimo e di flussi massimi converr� lavorare sul planar embedding. E forse anche per osservare che il grafo modificato non � planare. Il certificato � la suddivisione del $K_{3,3}$ esibita in figura. \textbf{(1pt)}

\begin{center}
\mbox{
  \psfig{figure=notplanar.pdf, height=7.0 true cm}
}
\end{center}

\begin{Nota}
Uno degli errori pi� comuni commessi dagli studenti � quello di sbagliare il certificato di planarit� del grafo. Vi propongo un esempio di tale errore: 
\begin{center}
\mbox{
  \psfig{figure=wrongplanarity.pdf, height=7.0 true cm}
}
\end{center}
In questo caso l'errore � stato quello di scegliere dei percorsi, che collegano i nodi \emph{'cerchiati'} a quelli \emph{'quadrettati'}, in modo tale da condividere alcuni nodi (che nella figura sono evidenziati con triangoli viola) tra loro, come per i percorsi che dal nodo \emph{x} portano ai nodi \emph{h} e \emph{d} che condividono i nodi \emph{'triangolati'} in viola \emph{a}, \emph{c}. 
\end{Nota}

\textbf{(1pt)} Un albero dei cammini minimi dal nodo \emph{s} a tutti gli altri nodi del grafo � rappresentato in figura dagli archi in linea spessa (sia tratteggiata che continua).

\begin{center}
\mbox{
  \psfig{figure=path.pdf, height=7.0 true cm}
}
\end{center}

\begin{Nota}
Vi rimando alle note teoriche sui Grafi o comunque a guardare con attenzione il funzionamento dell'algoritmo di Dijkstra.
\end{Nota}

\textbf{(1pt)}  Ovviamente ogni arco del grafo non contenuto nell?albero dei cammini minimi (ossia ogni arco in linea non spessa) pu� essere rimosso senza allontanare alcun nodo dal nodo \emph{s}. Inoltre, anche i tre archi in linea spessa ma tratteggiata possono essere rimossi poich� sostituibili con altri archi (sempre in linea tratteggiata). Ci sono quindi $2^{3} = 8$\footnote{tale numero si ottiene elevando 2 ad una potenza pari al numero di archi tratteggiati in figura, in questo particolare caso 3} alberi di cammini minimi dal nodo \emph{s}: sono ottenuti aggiungendo all'albero fornito sopra un qualsiasi sottoinsieme dell'insieme di archi in linea tratteggiata e rimuovendo quegli archi in linea spessa che si trovino ad entrare in nodi dove ora entra un arco in linea tratteggiata. \\

\textbf{(2+1 pt)}  La seguente figura esprime la famiglia degli alberi ricoprenti di peso minimo.

\begin{center}
\mbox{
  \psfig{figure=tree.pdf, height=7.0 true cm}
}
\end{center}

\begin{Nota}
Non esiste un unico modo per ottenere tali alberi, quindi vi mostro in breve quello che ho usato io. Ricordandovi che il vostro scopo � quello di toccare tutti i nodi del grafo, il procedimento � il seguente:
\begin{itemize}
\item per prima cosa evidenziate tutti gli archi di peso inferiore tra tutti (solitamente hanno peso 1 o 2), nel nostro caso sono gli archi di peso 1.\\
\item poi si procede con quelli immediatamente pi� grandi (in questo caso quelli di peso 2); facendo attenzione ai casi di scelta che bisogna tratteggiare, ad esempio un caso di scelta sono gli archi \emph{eb, bd, ed}. Sono una scelta perch� dal nodo \emph{e} si pu� arrivare sia al nodo \emph{b} che al nodo \emph{d} con un arco di peso 2, e analogamente si pu� arrivare a \emph{d} da \emph{b} con un altro arco di peso 2. \\
\item e si procede cos� fino ad esaurimento nodi.
\end{itemize}
\end{Nota}
Ci sono $2 \cdot 3 \cdot 8 = 48$\footnote{tale numero � dato dal prodotto delle scelte} alberi ricoprenti di perso minimo e ciascuno di essi include i $14$ archi in linea spessa e nera, pi� $2$ qualsiasi dei $3$ archi di peso $2$ in linea spessa e sfumata ($3$ scelte), pi� $1$ qualsiasi dei $2$ archi di peso $3$ in linea spessa e sfumata ($2$ scelte), pi� $2$ dei $5$ archi di peso $5$ ed in linea sfumata spessa, come da $8$ possibili scelte che possono essere meglio comprese cancellando tutti gli archi di peso maggiore di $5$ e contraendo tutti gli archi di peso inferiore a $5$ come illustrato nella seguente figura.

\begin{center}
\mbox{
  \psfig{figure=tree2.pdf, height=7.0 true cm}
}
\end{center}

\textbf{(1pt)} Un albero di peso minimo � il seguente: contiene tutti gli archi in linea nera grossa, pi� gli archi \emph{be} e \emph{de}, l'arco \emph{zy}, e gli archi \emph{bu} e \emph{vu}.
 
\textbf{(4pt)} L'arco \emph{dk}, appartiene a qualche soluzione ottima poich� di peso minimo $(5)$ nel taglio costituito dagli archi: \emph{ax, ac, fc, dc} e \emph{dk} \textbf{(1pt)}; non appartiene tuttavia a tutte le soluzioni ottime poich� � arco di peso massimo $(5)$ nel ciclo costituito dagli archi: \emph{db, bu, uv, vq, qk, kd} \textbf{(1pt)}. L'arco \emph{ax}, non appartiene ad alcuna soluzione ottima poich� unico arco di peso massimo $(6)$ nel ciclo costituito dagli archi: \emph{ax, xh, hc, ca} \textbf{(1pt)}. L'arco \emph{es}, appartiene a tutte le soluzioni ottime poich� unico arco di peso minimo $(5)$ nel taglio costituito dagli archi: \emph{es, fz} \textbf{(1pt)}.

\begin{Nota}
Per maggiori spiegazioni su quest'ultimo punto vi rimando alle note teoriche sui Grafi. 
\end{Nota}

\textbf{(1pt)} La seguente figura esibisce un flusso massimo (non esibisco tutti i passaggi che ho dovuto compiere per ottenerlo) ed un taglio (minimo) che ne dimostra l'ottimalit�.

\begin{center}
\mbox{
  \psfig{figure=flusso.pdf, height=7.0 true cm}
}
\end{center}
\begin{Nota}
Fate attenzione che il valore del taglio deve avere lo stesso valore del flusso, cio� la somma del peso degli archi che tagliate deve essere uguale al flusso.
\end{Nota}
Il flusso ha valore $14$ e satura l'insieme degli archi (due soli archi) che attraversano la curva sfumata portandosi dal lato di \emph{s} al lato di \emph{t}. Questi $2$ archi costituiscono pertanto un minimo \emph{s,t}-taglio, anch'esso di valore $14$ e che certifica pertanto l'ottimalit� del flusso proposto. \textbf{(1pt)} (Sia il flusso che il taglio sarebbero stati pi� immediati a vedersi e verificarsi nel planar embedding. Puoi provare a rappresentarteli l�).
\textbf{(1pt)} Il massimo flusso da \emph{s} a \emph{q} a valore $9$ e la stella di \emph{q} � un taglio che ne certifica l'ottimalit�. \textbf{(1pt)} A parte questa strozzatura, vi � altrimenti ampio margine nell'inviare flusso da \emph{s} a \emph{q} ed evitiamo quindi di fornire descrizione di una tale soluzione ammissibile di valore $9$ (cosa che ovviamente voi non potete mai fare: se volete totalizzare i rispettivi punti dovete innanzitutto fornirmi le soluzioni/certificati!). Essa pu� comunque essere facilmente prodotta con ovvie modifiche partendo dal flusso di valore $14$ dato in figura.
  
\begin{Nota}
L'ultimo consiglio che mi sento di darvi � quello di fare quanti pi� esercizi potete perch� alcuni di questi punti vi verranno meglio con la pratica.
\end{Nota}
\medskip
 
\end{document}