\documentclass[11pt]{article}
\usepackage[italian]{babel}


\textwidth 15.5cm
\textheight 21.5cm
\topmargin 0cm
\evensidemargin 0in
\oddsidemargin 0in


\def\bbbr{{\rm I\!R}}
\def\bbbm{{\rm I\!M}}
\def\bbbn{{\rm I\!N}}

\def\sm{\setminus}
\newcommand {\ol}[1]{\overline{#1}}

\newtheorem{Teo}{Theorem}[section]
\newtheorem{Con}[Teo]{Conjecture}
\newtheorem{Que}{Question}
\newtheorem{Cor}[Teo]{Corollary}
\newtheorem{Property}[Teo]{Property}
\newtheorem{Pro}[Teo]{Proposition}
\newtheorem{Cla}[Teo]{Claim}
\newtheorem{Lem}[Teo]{Lemma}
\newtheorem{Obs}[Teo]{Observation}

\pagestyle{empty}
 
\begin{document}

\begin{center}
   {\LARGE \bf Ma questo non \`e che un semplice problema di PL !!!}
\end{center}

\medskip

{\Large \bf Formulare i seguenti problemi secondo il modello della PL}\\


{\large \bf Problemi matematici per agricoltori}\\

{\sc \bf Problema~1\/:}
Un contadino dispone di 2 ettari di terreno.
Il contadino non pu\`o dedicare pi\`u di 5 mesi l'anno
alla cura dei suoi campi.
Un ettaro coltivato a Renette Golden gli richiede
3 mesi di lavoro mentre un ettaro coltivato
a Canada richiederebbe solamente 2 mesi di lavoro all'anno.
Tuttavia un ettaro coltivato a Renette Golden frutterebbe
5 soldi ogni anno contro i 4 soldi ricavabili dallo stesso
ettaro se coltivato a Canada.
Dovendo decidere come impiantare il terreno, 
quale politica consentirebbe al contadino
di massimizzare il suo guadagno?
Formulare il problema secondo il modello della PL.

\bigskip

{\sc \bf Problema~2\/:}
Una fattoria consiste di due lotti di terreno A e B
rispettivamente di 200 e 400 acri.
Sei tipi di cereali, numerati da 1 a 6,
possono esservi coltivati.
Per ogni quintale di cereale il profitto \`e dato dalla
seguente tabella:

\begin{table}[!htb]
%\hspace*{-1.5cm}
%\begin{minipage}[t]{15cm}
%\begin{footnotesize}
\begin{tabular}{|l|cccccc|} \hline
   cereale    &  1 &  2 &  3 &  4 &   5 &  6 \\
 \hline
   profitto/q & 48 & 62 & 28 & 36 & 122 & 94 \\ 
\hline

\end{tabular}
%\end{footnotesize}
%\end{minipage}
%\caption{}
%\label{}
\end{table}

Ogni quintale di cereale necessita
di una certa area (in acri) e di una certa
quantit\`a d'acqua (in  mc) secondo la seguente tabella:

\begin{table}[!htb]
%\hspace*{-1.5cm}
%\begin{minipage}[t]{15cm}
%\begin{footnotesize}
\begin{tabular}{|l|cccccc|} \hline
   cereale            &    1 &     2 &     3 &     4 &    5 &     6 \\
 \hline
   area su A (acri/q) & 0.02 &  0.03 &  0.02 & 0.016 & 0.05 &  0.04 \\ 
   area su B (acri/q) & 0.02 & 0.034 & 0.024 &  0.02 & 0.06 & 0.034 \\ 
   acqua       (mc/q) &  120 &   160 &   100 &   140 &  215 &   180 \\ 
\hline

\end{tabular}
%\end{footnotesize}
%\end{minipage}
%\caption{}
%\label{}
\end{table}


Il volume totale di acqua disponibile
\`e di 400000 mc.
Si vuole massimizzare il profitto.
Formulare il problema secondo il modello della PL.\\

Il secondo problema \'e tratto dal Capitolo~3
del libro ``Elementi di Programmazione Matematica'' 
di F.~Maffioli.\\

\bigskip


{\large \bf Il Problema della Dieta}\\

{\sc \bf Problema~3\/:}
Per ben alimentare un certo tipo di animale
\`e necessario fornirgli 4 sostanze base:
A, B, C e D.
La quantit\`a minima giornaliera che ogni animale richiede
\'e data da: 0,4 kg di A; 0,6 kg di B; 2 kg di C;
1,3 kg di D.
Il foraggio \`e ottenuto mescolando due farine M ed N.\\

1kg di M contiene: 100 gr di A; nulla di B; 100 gr di C; 100 gr di D.\\
\indent
1kg di N contiene: nulla di A; 100 gr di B; 200 gr di C; 100 gr di D.\\

Con 1 ECU possiamo comperare 4 kg di M o 8 Kg di N.
Formulare il problema di minimizzare i costi
secondo il modello della PL
e risolverlo per via geometrica e con il metodo del simplesso.\\

Tratto dal Capitolo~3
del libro ``Elementi di Programmazione Matematica'' 
di F.~Maffioli.\\

\bigskip


{\large \bf Il Problema del Trasporto}\\

{\sc \bf Problema~4\/:}
Una ferriera possiede due miniere e tre impianti
per la produzione di acciaio.
Gli impianti richiedono rispettivamente 70, 140 e 100
tonnellate di minerale,
mentre le miniere producono nello stesso periodo
rispettivamente 100 e 200 tonnellate di minerale.
La tabella seguente riporta i costi di trasporto
dalle miniere agli impianti in lire/tonnellate.

\begin{center}
%\hspace*{-1.5cm}
%\begin{minipage}[t]{15cm}
%\begin{footnotesize}
\begin{tabular}{|c|c|c|c|} \hline
   & \multicolumn{3}{|c|}{all'impianto} \\
\hline
   dalla miniera & 1 & 2 & 3 \\
\hline
   1 & 100 & 160 & 250 \\
   2 & 150 & 300 & 200 \\ 
\hline

\end{tabular}
%\end{footnotesize}
%\end{minipage}
%\caption{}
%\label{}
\end{center}

Formulare il problema di minimizzare
il costo di trasporto
secondo il modello della PL.\\

Tratto dal Capitolo~3
del libro ``Elementi di Programmazione Matematica'' 
di F.~Maffioli.\\

\bigskip


{\large \bf Interpolazione lineare di dati sperimentali}\\

{\sc \bf Problema~5\/:}
Di una grandezza $y$ dipendente da una variabile $x$
si sono effettuate 6 misure riportate come coppie $(x,y)$
nell'elenco seguente:
$(1,16)$ $(2,18)$ $(3,12)$ $(4,7)$ $(5,6)$ $(6,6)$.
Si vuole determinare la retta del piano cartesiano
che ``meglio'' interpola le sei misure
nel senso di minimizzare il massimo scarto 
tra valore misurato e valore della retta.
Modellare come un problema di programmazione lineare.\\  

Pi\`u in generale,
nell'attivit\`a sperimentale
accade di considerare il seguente problema.
Dato un sistema di equazioni
\[
   \sum_{j=1}^n a_{ij} = b_i \mbox{\hspace{1cm}} i = 1, \ldots, m
\]

con $m > n$,
trovare la ennupla di numeri $\bar{x}_1, \ldots, \bar{x}_n$
che ne costituisce la ``miglior'' approssimazione.
Tale concetto pu\`o precisarsi meglio definendo con
\[
   \epsilon_i = \left| \sum_{j=1}^n a_{ij} - b_i\right|
\]
 
l'{\em errore} con cui l'ennupla data soddisfa
alla $i$-esima equazione, $i = 1, \ldots, m$.
Criteri usati sono quelli dell'errore peggiore ($\max_i e_i$),
dell'errore medio ($\sum_{i=1}^m e_i$)
o dell'errore quadratico medio ($\sum_{i=1}^m {e_i}^2$).
Quest'ultimo caso non pu\'o essere inquadrato come un
problema di PL.
Mostrare come gli altri due casi possano
essere ricondotti al modello della PL.\\


Tratto dal Capitolo~3
del libro ``Elementi di Programmazione Matematica'' 
di F.~Maffioli.\\

\bigskip

{\large \bf Gestione di Produzione}\\

{\sc \bf Problema~6\/:}
Il gestore di una raffineria dispone di 10 milioni
di barili di greggio di tipo A
e di 6 milioni di greggio di tipo B.
La raffineria dispone di tre diversi impianti
per produrre sia nafta per riscaldamento (profitto 3 kL/barile)
sia benzina (5 kL/barile)
con le caratteristiche di rendimento (in barili)
riportate in figura:

\begin{table}[!htb]
%\hspace*{-1.5cm}
%\begin{minipage}[t]{15cm}
%\begin{footnotesize}
\begin{tabular}{|c|c|c|c|c|} \hline
   & \multicolumn{2}{|c|}{Input} & \multicolumn{2}{|c|}{Output} \\
\hline
   Impianto & A & B & Benzina & Nafta \\
\hline
   1 & 3 & 5 & 4 & 3 \\
   2 & 1 & 1 & 1 & 1 \\ 
   3 & 5 & 3 & 3 & 4 \\ 
\hline

\end{tabular}
%\end{footnotesize}
%\end{minipage}
%\caption{}
%\label{}
\end{table}

Formulare il problema di massimizzare il profitto
secondo il modello della PL.\\

Tratto dal Capitolo~3
del libro ``Elementi di Programmazione Matematica'' 
di F.~Maffioli.\\

\bigskip


{\sc \bf Problema~7\/:}
La GELAR produce pacchi di patatine surgelate,
sia a bastoncino (A),
che in pezzi pi\`u piccoli (B)
(per le cosidette patate alla svizzera),
e di fiocchi (C) (non surgelati) per il pur\`e.
La compagnia acquista patate da due produttori (p1 e p2)
con rese differenti,
riportate nella seguente tabella
(l'avanzo del 30% per entrambi i produttori
\`e lo scarto non recuperabile).

%\begin{table}[!htb]
%\hspace*{-1.5cm}
%\begin{minipage}[t]{15cm}
%\begin{footnotesize}
\begin{center}
\begin{tabular}{|c|ccc|} \hline
produttore & A & B & C \\
\hline
   p1 & 20\% & 20\% & 30\% \\
   p2 & 30\% & 10\% & 30\% \\ 
\hline

\end{tabular}
\end{center}
%\end{footnotesize}
%\end{minipage}
%\caption{}
%\label{}
%\end{table}

Il profitto della GELAR \`e di 5 lire
per etto di patate provenienti dal produttore 1
e di 6 lire per etto per quelle provenienti dal produttore 2.
Ci sono poi delle limitazioni alla quantit\`a di ciascun
tipo di prodotto: 6 tonnellate di A, 4 di B e 8 di C.
Formulare il problema di massimizzare il profitto
secondo il modello della PL
e risolverlo per via geometrica e con il metodo del simplesso.\\

Tratto dal Capitolo~3
del libro ``Elementi di Programmazione Matematica'' 
di F.~Maffioli.\\

\bigskip

{\large \bf Gestione di Personale}\\

{\sc \bf Problema~8\/:}
Un'impresa per la produzione di beni di consumo
deve essere gestita in modo da tenere conto delle
fluttuazioni della domanda di tali beni su un periodo
standard di 6 mesi (gennaio -- giugno e luglio -- dicembre).
I mezzi per adattare l'azienda
a tali fluttuazioni sono:
\begin{itemize}
   \item[1.] cambiare la quantit\`a di forza lavoro assumendo
         o licenziando operai;
   \item[2.] coprire le punte della domanda con lavoro straordinario;
   \item[3.] immagazzinare merce in vista di richieste future.
\end{itemize}

Ognuna di queste strategie ha le sue limitazioni.
La strategia (1) \`e limitata ad un massimo di 5 operai al mese
(sia in pi\`u che in meno) con un sovrapprezzo
di 500 kL per una nuova assunzione e di 700 kL per ogni licenziamento.
La strategia (2) deve tener conto che ogni operaio pu\`o al massimo
fornire in pi\`u 6 unit\`a al mese col lavoro straordinario
mentre ne produce 20 regolarmente e che tale prestazione
graver\`a per 30 kL in pi\`u per ogni unit\`a di merce prodotta.
Ci sono attualmente (inizio semestre)
40 operai ed il magazzino \`e vuoto:
di fatto la politica a lungo
termine dell'azienda richieda che il magazzino sia vuoto
alla fine di ogni semestre.
Le fluttuazioni delle domande previste nei prossimi
6 mesi sono riportate in figura.

\begin{center}
%\begin{table}[!htb]
%\hspace*{-1.5cm}
%\begin{minipage}[t]{15cm}
%\begin{footnotesize}
\begin{tabular}{||l|cccccc||} \hline \hline
   mese & 1 & 2 & 3 & 4 & 5 & 6 \\
\hline
   unit\`a di merce richiesta & 700 & 600 & 500 & 800 & 900 & 800 \\
\hline
\end{tabular}
%\end{footnotesize}
%\end{minipage}
%\caption{}
%\label{}
%\end{table}
\end{center}

Formulare come un problema di PL.\\

Tratto dal Capitolo~3
del libro ``Elementi di Programmazione Matematica'' 
di F.~Maffioli.\\

\bigskip


Buon Lavoro!

\end{document}







