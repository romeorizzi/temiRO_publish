\documentclass[a4paper,11pt]{article}
\usepackage[italian]{babel}
\usepackage{fullpage}
\usepackage[utf8]{inputenc}
\pagenumbering{arabic}
\usepackage[nointegrals]{wasysym}
\usepackage{graphicx}
\usepackage{mathtools,amssymb,amsfonts}
\usepackage{booktabs}
\usepackage{tabbing}
\date{}


\begin{document}
	
	\section*{Interpretazione economica delle variabili duali}
	
	Per molti problemi di Programmazione Lineare (PL)
        che derivano dalle applicazioni,
        alle variabili del problema duale possono essere attribuite delle interpretazioni significative e feconde.
        Questa interpretazione \`e la porta per dare concretezza al problema duale, controparte dialettica ed alter ego intimo del problema originalmente formulato nella veste di problema primale. Importanti sviluppi nell'economia (macroeconomia e teoria dell'equilibrio economico) hanno preso avvio dallo studio di questa intima relazione.\\
        
        Anche se si \`e soliti parlare di interpretazione economica delle variabili duali (spesso il titolo di uno dei capitoli entro un libro di PL),
        in verit\`a l'interpretazione di cui risulti pi\`u opportuno rivestire le variabili duali dipende dal problema formulato e dal suo contesto applicativo.
        Esistono delle regole meccaniche che consentono di produrre il problema duale di un dato problema di PL, e queste regole non dipendono in alcun modo dal significato rivestito dalle variabili primali.
        Occorre quindi restituire semantica al problema duale cui si perviene per mera applicazione di queste regole meccaniche. 
        In generale, una prima indicazione sul modo in cui queste variabili duali dovrebbero essere interpretate viene da un argomento euristico che viene usato occasionalmente in fisica elementare ad è conosciuto come ``analisi dimensionale''.
        Illustreremo questo modo di procedere su due casi paradigmatici. Il primo parte da un problema primale in forma standard rivestito da un'iterpretazione importante e di ampia valenza. Questo problema costituisce probabilmente il modello di PL p\`u importante nella applicazioni: ``il problema di gestire la produzione massimizzando i profitti''. Il secondo parte da un problema primale in forma duale standard, rivestito da un'interpretazione che costituisce il celebre e rilevante modello noto come ``il problema della dieta''.\\

	\section*{Pianificare la produzione massimizzando i profitti entro il limite delle risorse}
      
        In questa sezione trattiamo di un problema primale in forma standard,
        ossia:
	
	\begin{equation}\begin{split}
	&\text{max}\qquad\qquad\sum_{j=1}^{n}c_jx_j\\
	&\text{soggetto a} \qquad\begin{split} &\sum_{j=1}^{n}a_{ij}x_j \leq b_i \qquad(i=1,2,\ldots,m)\\
	&x_j\geq0 \qquad(j=1,2,\ldots,n),\end{split}
	\end{split} 
	\end{equation}

        Per procedere con questo esempio e caso paradigmatico
        promuoviamo il freddo problema matematico a modello
        rivestendolo della seguente semantica e contestualizzazione.
        Si veda~(1) come il problema di massimizzare il profitto in una fabbrica (volendo essere pi\`u concreti, anche se non necessario contestualizzare oltre, si pensi ad una fabbrica che produce mobili). Ciascuna variabile primale $x_j$ indica la quantità di $j$-esima tipologia di prodotto (ad esempio scrivanie o sedie) da prodursi, ogni $b_i$ specifica la disponibilità della $i$-esima risorsa (come legno o metallo).
        Abbiamo concretizzato il problema, ora esso ha un suo sapore specifico.
 
        Possiamo partire ora con l'analisi dimensionale applicandola per meglio afferrare la natura dei vari elementi, prima del problema primale, poi anche quelli del problema in forma duale.
	Notiamo che ogni $a_{ij}$ è espresso in unità di risorsa $i$ per unità di prodotto $j$ (infatti, ogni $a_{ij}$ è l'ammontare di risorsa $i$ richiesta per realizzare una unità di prodotto $j$) e che ogni $c_j$ è in dollari per unità di prodotto $j$ (infatti, ogni $c_j$ è il profitto {\bf netto} che deriva dalla produzione di un'unità di prodotto $j$). \`E molto opportuna questa accortezza di aver elaborato i dati in modo che i coefficienti $c_j$ si riferiscano ai profitti \emph{netti}. In particolare, anche se delle varie risorse abbiamo a disposizione quantit\`a prefissate come specificate dalle $b_i$, ci\`o non significa che quelle risorse non abbiano un valore di mercato (non a caso la teoria dell'equilibrio economico postula si possa sempre vendere od acquistare un bene al suo valore di mercato) ed i coefficienti $c_j$ devono tener conto dei valori di mercato delle risorse impiegate e sottrarli ai meri realizzi alla vendita.\\

        Il duale di un problema in forma standard come quello sopra
        \`e il seguente problema in forma duale standard
        nelle variabili $y_1,y_2,\ldots,y_m$ (una per ogni vincolo del primale.)

        \begin{equation}\begin{split}
	&\text{min}\qquad\qquad\sum_{i=1}^{m}b_iy_i\\
	&\text{soggetto a} \qquad\begin{split} &\sum_{i=1}^{m}a_{ij}y_i \geq c_j \qquad(j=1,2,\ldots,n)\\
	&y_i\geq0 \qquad(i=1,2,\ldots,m),\end{split}
	\end{split} 
	\end{equation}

        Il problema duale trova giustificazione come tentativo di produrre degli upper bounds al valore ottimo del primale.
        Assumiamo il lettore abbia gi\`a fatto conoscenza della teoria della dualit\`a.
        
	Nel problema duale, per rendere confrontabili la parte sinistra, $\sum a_{ij}y_i$, e la parte destra $c_j$, dobbiamo esprimere ogni $y_i$ in dollari per unità di risorsa $i$. In questo modo, si è portati a sospettare che ciascun $y_i$ misuri il valore unitario della risorsa $i$-esima. Il teorema esposto più sotto dà corpo e sostanza a questo sospetto.\\
        
	\textbf{Teorema.} (degli scarti complementari).
		Siano $\tilde{x}$ e $\tilde{y}$ rispettivamente una soluzione primale estesa ammissibile e una soluzione duale estesa ammissibile. Allora $\tilde{x}$ ed $\tilde{y}$ sono ottime per i rispettivi problemi se e solo se 
		\begin{eqnarray*}
			x_jy_{m+j}=0 & j = 1,\ldots,n \\
			x_{n+i}y_i=0 & i = 1,\ldots,m
		\end{eqnarray*}
		o in termini vettoriali 
		\begin{equation*}
		\tilde{x} \bot \tilde{y}
		\end{equation*}
\\
	\textbf{Lemma.} Se (1) ammette almeno una soluzione di base ottima non degenere
	allora il suo duale ammette un'unica soluzione ottima.\\
        \textbf{Dimostrazione.}
           Che il duale ammetta almeno una soluzione ottima segue dal teorema della dualit\`a forte. L'unicit\`a disegue invece dal teorema degli scarti complementari considerando che la non-degenerativit\`a della soluzione primale implica che gli $m$ termini noti nel dizionario as essa associato siano tutti non nulli, imponendo $m$ condizioni indipendenti su $y$. \Square \\
	\\
	\\
	\textbf{Teorema.} Sia x una soluzione di base ottima non degenere di (1), ed y la soluzione ottima del duale. Esiste allora un valore $\epsilon > 0$ tale che: se $|t_i|\leq \epsilon$ per ogni $i=1,2,\ldots,m$, allora il problema  
		\begin{equation}\begin{split}
	&\text{max}\qquad\qquad\sum_{j=1}^{n}c_jx_j\\
	&\text{soggetto a} \qquad\begin{split} &\sum_{j=1}^{n}a_{ij}x_j \leq b_i + t_i \qquad(i=1,2,\ldots,n) \\
	&x_j\geq0 \qquad(j=1,2,\ldots,n)\end{split}
	\end{split}\end{equation}
	ha una soluzione ottima e il suo valore ottimo vale
	$$z^* + \sum_{i=1}^{m}y^*_it_i$$
	con $z^*$ valore ottimo di (1) e con $y^*_1,y^*_2,\ldots,y^*_m$ soluzione ottima del corrispondente duale.\\
	\\
	\textbf{Dimostrazione.} \textquotedblleft$ \leq $\textquotedblright : Se B è una soluzione di base ottima per (1), tale che $y^*=(c^TB^{-1})^T$ è la soluzione ottima del duale di (1), il valore ottimo è $z^*=c^TB^{-1}b$. \\Per il teorema della dualità forte, $y=(c^TB^{-1})^T$ è una soluzione ammissibile del duale e $by=y^Tb=z^*$ e y è soluzione ammissibile del duale di (2), poichè non sono stati modificati nè gli $a_{ij}$ nè i $c_j$.
	Il valore della funzione obbiettivo di $y$ nel duale di (2) è $$(b+t)y=by+ty=z^*+\sum_{i=1}^{m}y^*_it_i $$ e per dualità debole ogni soluzione del primale non può avere valore maggiore.\\
	\textquotedblleft$\geq$\textquotedblright: La soluzione di base di (2) è $x=B^{-1}(b+t)$ e il valore della funzione obbiettivo è $$cx=(c^TB^{-1})(b+t)=c^TB^{-1}b+c^TB^{-1}t=z^*+\sum_{i=1}^{m}y^*_it_i .$$ Quindi, avendo soluzioni ammissibili di (2) e poichè il primale (2) ha lo stesso valore di funzione obbiettivo del corrispondente duale, la soluzione è ottima.
	\\
	Il fatto che esista $\epsilon>0$ segue dall'ipotesi di non degeneratività della soluzione di base ottima. \Square \\
\\ Questo teorema rivela gli effetti di piccole variazioni nei rifornimenti di risorse sul profitto totale netto della fabbrica. Con ogni unità extra di risorsa $i$, il profitto incrementa di $y^*_i$ dollari. 
	Quindi, $y^*_i$ specifica il massimo ammontare che l'azienda dovrebbe essere disposta a pagare, oltre all'attuale prezzo di scambio, per ogni unità extra di risorsa $i$. Per questo motivo, $y^*_i$ è spesso chiamata \textit{valore marginale} dell'$i$-esima risorsa, l'aggettivo \textquotedblleft marginale" si riferisce alla differenza tra il prezzo di scambio e il valore attuale della risorsa. Un altro termine comune usato per $y^*_i$ è \textit{prezzo ombra} della risorsa $i$-esima.\\
	
\section*{Esempio 1}	Per chiarire queste osservazioni, immaginiamo un boscaiolo che ha a disposizione 100 acri di latifoglie. L'abbattimento degli alberi e la rigenerazione dell'area costa \$10 per acre in risorse immediate e porta un profitto successivo di \$50 per acre.
	In alternativa, si potrebbero abbattere le latifoglie e piantare nell'area dei pini: questo costerebbe \$50 per acre e un ritorno successivo di \$120 per acre. Dunque, il profitto netto risultante dai due trattamenti è di \$40 e \$70 per acre, rispettivamente. Sfortunatamente, il trattamento che porta più profitti non può essere applicato all'intera area dato che solo \$4000 sono disponibili per affrontare i costi immediati. Il problema del boscaiolo è quindi 
		\begin{equation}\begin{split}
	&\text{max}\qquad\qquad\ 40x_1+70x_2\\
	&\text{soggetto a} \qquad
	\begin{split} 
	 x_1+x_2&\leq 100\\
	10x_1+50x_2&\leq 4000\\
	x_1,x_2&\geq0.
	\end{split}
	\end{split}
	\end{equation}
	\\
	La soluzione ottima è $x^*_1=25$ e $x^*_2=75$. Quindi, il boscaiolo dovrebbe disboscare l'intera area, lasciare che 25 acri si rigenerino e piantare i rimanenti 75 acri con dei pini. In accordo con questo programma, l'investimento iniziale di \$4000 porta un profitto netto di \$6250.
	Evidentemente, il capitale iniziale rappresenta una risorsa importante. Infatti, il boscaiolo potrebbe ricevere il consiglio di incrementare il livello di questa risorsa chiedendo un prestito a breve tempo; il risultante extra profitto potrebbe anche portare un tasso di interesse svantaggioso. Per esempio, sopponiamo che il boscaiolo possa farsi prestare \$100 ora e restituire \$180 in seguito; lo dovrebbe fare? Oppure, potrebbe essere tentato di spostare i suoi \$4000 verso altre aziende. Per esempio, supponiamo che potrebbe investire oggi \$100 e guadagnare \$180, dovrebbe farlo?
	In accordo con il teorema, la risposta è nascosta nella soluzione ottima \\
	$y^*_1=32.5, y^*_2=0.75$
	del problema duale: il boscaiolo dovrebbe  chiedere un prestito se e solo se l'interesse fosse più basso di 75 cent per dollaro e dovrebbe fare piccoli investimenti se e solo se il profitto fosse maggiore di 75 cent per dollaro.
	Queste affermazioni, le quali validità sono garantite dal teorema, sono facili da giustificare direttamente. Avendo preso in prestito $t$ dollari, il boscaiolo mira a
			\begin{equation}\begin{split}
	&\text{max}\qquad\qquad\ 40x_1+70x_2\\
	&\text{soggetto a} \qquad
	\begin{split} 
	x_1+x_2&\leq 100\\
	10x_1+50x_2&\leq 4000+t\\
	x_1,x_2&\geq0.
	\end{split}
	\end{split}
	\end{equation}
	
	Ogni soluzione ammissibile $x_1, x_2$ di questo problema soddisfa le disuguaglianze
	\begin{equation}
	40x_1+70x_2=32.5(x_1+x_2)+0.75(10x_1+50x_2)\leq3250+0.75(4000+t)=6250+0.75t
	\end{equation}
	e quindi l'extra profitto non eccederà mai 0.75t. Infatti, se $t\leq1000$, allora il boscaiolo può realizzare un profitto addizionale di $0.75t$ ponendo
	\begin{equation}
	x_1=25-0.025t, x_2=75+0.025t.
	\end{equation}
	Gli investimenti in altre aziende possono portare a valori negativi di $t$ in (4); come risultato di un tale investimento il profitto netto dall'azienda originale diminuisce. Se $-t$ dollari sono usati per altri investimenti ($-t$ è positivo!) allora, per (5) il profitto dall'azienda che abbatte le latifoglie calerà di $0.75(-t)$ o anche più. Infatti, se $-t\leq3000$, allora il calo può essere limitato a solo $0.75(-t)$ scegliendo $x_1$ e $x_2$ in accordo con (6).
	Forse dovrebbe essere enfatizzato il fatto che il teorema ha a che fare con piccoli cambiamenti $t_i$ nel livello della risorsa; la sua conclusione può fallire quando i $t_i$ sono grandi. Per esempio, il nostro boscaiolo non può usare prestiti superiori a \$1000 e, potrebbe voler investire tutti i suoi \$4000 in un'altra impresa, oppure ricevere un consiglio sbagliato e chedere solo 75 cents di profitto per ogni dollaro. (Una parte del Teorema può essere usata anche se i $t_i$ sono grandi).
\\

Ora supponiamo che il boscaiolo abbia una nuova possibilità che prima non era disponibile, abbattere le latifoglie e piantare nell'area delle conifere.
Per fare una veloce valutazione dell'attività, il boscaiolo potrebbe far ricorso al valore marginale delle sue risorse: \$32.5 per acre di latifoglia e \$0.75 per dollaro di capitale. Se la nuova attività richiede $a$ dollari per acre, allora le risorse consumate da questa attività per acre sono valutate come \$$(32.5+0.75a)$ e quindi vale la pena considerare l'attività se e solo se il suo profitto netto per acre supera questo schema. 
\\

	\section*{Problema della dieta}

        Come il problema di pianificazione della produzione costituisce l'indiscusso archetipo per l'interpretazione economica di problemi di PL con primale in forma primale standard, cos\`\i\ il problema della dieta costituisce un possibile riferimento per l'interpretazione economica di problemi con primale in forma duale standard.

        Il problema della dieta \`e il seguente:
        \begin{quote}
        un allevatore deve preparare una miscela alimentare per il suo bestiame garantendo un apporto di $b_i$ unità per ogni sostanza nutritiva
 (vitamine, proteine, carboidrati etc.).  Il mercato mette a disposizione degli alimenti, corrispondenti ai mangimi di diverse case produttrici:  ogni unità di alimento $j$ ha un costo $c_j$ e un apporto di $a_{ij}$ unità della sostanza nutritiva $i$.  L’allevatore vuole determinare le
        quantità di alimenti da acquistare per soddisfare a costo minimo i requisiti di sostanze nutritive.
        \end{quote}
        Se abbiamo $n$ alimenti e $m$
 sostanze nutritive, il problema della dieta,
 dal punto di vista dell’allevatore, è formulabile con il seguente modello di programmazione lineare.\\
 Parametri:
 \begin{itemize}
 \item $c_j$ costo unitario alimento $j (j=1,\ldots,n)$
 \item $a_{ij}$ tasso  di  presenza  della  sostanza $i (i= 1,\ldots,m)$  nell’alimento
 $j$
 (ad  es.   grammi  per
 unità di alimento
 \item $b_j$ fabbisogno minimo sostanza $i$ (ad es. in grammi).
\end{itemize}
Variabili decisionali
\begin{itemize}
	\item $x_j$ quantità dell'alimento $j$ da acquistare.
\end{itemize}	
Modello: 
\begin{equation}\begin{split}
&\text{min}\qquad\qquad\sum_{j=1}^{n}c_jx_j \qquad \text{(minimizzazione dei costi)}\\
&\text{soggetto a} \qquad\begin{split} &\sum_{j=1}^{n}a_{ij}x_j \geq b_i \qquad(i=1,2,\ldots,n)\qquad \text{(soddisfazione dei fabbisogni nutritivi)}\\
&x_j\geq0 \qquad(j=1,2,\ldots,n)\end{split}
\end{split}\end{equation}

Introducendo le variabili duali $y_i$ associate ad ogni vincolo per la sostanza nutritiva $i$, il corrispondente problema duale è:
\begin{equation}\begin{split}
&\text{max}\qquad\qquad\sum_{i=1}^{m}b_iy_i\\
&\text{soggetto a} \qquad\begin{split} &\sum_{i=1}^{m}a_{ij}y_i \leq c_j \qquad(j=1,2,\ldots,n) \\
&x_j\geq0 \qquad(i=1,2,\ldots,m)\end{split}
\end{split}\end{equation}

Possiamo vedere la funzione obiettivo come la massimizzazione di un profitto derivante
dalla  vendita  di $b_i$ unità  della  sostanza nutritiva $i$.  Le  variabili  duali $y_i$ rappresentano quindi  il  prezzo  di  vendita  unitario  della  sostanza $i$.  \\
 Il  problema  duale  può  essere  infatti  interpretato  considerando  un'azienda chimica  che  decide  di  immettere  sul
mercato i nutrienti puri (proteine,  vitamine,  carboidrati  etc.).   Essa vuole  determinare  il  prezzo  di  vendita
unitario di ciascun nutriente in modo da massimizzare il suo profitto. \\ La funzione obiettivo del problema duale utilizza i termini noti dei  vincoli  primali:  l’allevatore  acquista  la  quantità  di  sostanza
$i$ di  cui  necessita,  cioè $b_i$
unità al prezzo unitario $y_i$.  I vincoli del problema duale esprimono dei criteri di competitività dei prezzi $y_i$, rispetto all’attuale fornitore dell’allevatore:  ad esempio, affinchè
l’allevatore  decida  di  fornirsi  dal  nuovo  produttore,  è  sufficiente  che  il  costo  sostenuto per  ricostituire  tutti  gli  apporti  nutritivi  di  una  unità  della  miscela
$j$ a  partire  dalle
sostanze  acquistate  ($\sum a_{ij}y_i$)  non  sia  maggiore  del  costo  per  ottenere  gli  stessi  apporti acquistando direttamente il prodotto
$j$ (costo $c_j$).  Se non fossero rispettati questi
vincoli,  l’allevatore  potrebbe  preferire  continuare  ad  acquistare  i  prodotti  “compositi”.
\\
\section*{Esempio 2}
In un allevamento di polli
si usano per il mangime
due tipi
di cereali,
A
e
B
. Il mangime
dev
e soddisfare
certi
requisiti
nutritivi:
deve contenere
almeno
8 unità
a di carboidrati,
15
unità
a di proteine
e 3 unità
a di vitamine
per unità
a di peso.
Il contenuto
unitario
di
carboidrati,
proteine
e vitamine
ed il costo
unitario
di
A
e
B
sono
riportati
nella
seguente
tabella
insieme
ai requisiti
minimi
giornalieri.
\begin{center}
\begin{tabular}{l*{3}{c}}
	
	& A & B & min. giornaliero \\
	\midrule
	carboidrati & 5 & 7 & 8 \\
	proteine & 4 & 2 & 15 \\
	vitamine & 2 & 1 & 3 \\
	costo unitario & 1200 & 750 & \\
	
\end{tabular}
\end{center}

Siano $x_1$ ed $x_2$ rispettivamen
te il numero
di unità
a di cereale
$A$
e
$B$
impiegate
nel
mangime,
il numero di unità di carboidrati
presenti nel
mangime è dato
da
$5x_1+7x_2,$
e poichè il fabbisogno
minimo
di carboidrati
è di 8 unità, deve risultare
$5x_1+7x_2 \geq 8$; analogamente,
per le unità di proteine
deve risultare
$4x_1
+ 2x_2 \geq
15$
e per le unità di
vitamine
$2x_1
+
x_2 \geq 3.$ Ai
tre
vincoli
precedenti si devono
aggiungere
le ovvie
condizioni
di non
negatività
a delle
variabili,
$x_1
; x_2 \geq 0;$ infine
la funzione
obiettivo è $1200x_1+ 750x_2.$
La
dieta
di costo
minimo è data
dunque
da
una
soluzione
del
seguente problema
di
PL:

\begin{equation}\begin{split}
\text{min}\qquad\qquad\ 1200x_1+750x_2\\
\text{} \qquad
\begin{split} 
&65x_1+7x_2\geq 8\\
&4x_1+2x_2\geq 15\\
&2x_1+x_2 \geq 3 \\
&x_1,x_2\geq0.
\end{split}
\end{split}
\end{equation}

Al
problema
in esame è \textit{naturalmente}
associato
un
altro
problema
che chiameremo
il
problema
del
venditore di pillole
per polli: si tratta
di stabilire
i prezzi
di vendita
di
pillole
rispettivamente di carboidrati,
proteine
e vitamine
in modo
che
il ricavato
della
vendita
sia
massimo
e che i prezzi
siano
competitivi,
ossia
che l'allevatore
di polli
ritenga
non
svantaggioso
acquistare
le pillole
invece
dei
cereali
$A$
e
$B.$ Supponiamo
che
ciascuna
pillola
contenga
un'unità del
corrispondente elemento nutritivo, e siano
$y_1$; $y_2$ e $y_3$
i prezzi
rispettivamente di una
pillola
di carboidrati,
proteine
e vitamine:
poichè l'allevatore
deve
percepire
la dieta
a base
di pillole
non
più costosa
della
dieta
a base
di cereali
dovrà
risultare
$5y_1
+ 4y_2
+ 2y_3 \leq
1200$,
cioè  il costo
dell'equivalente (da
un
punto
di vista
nutritivo) in pillole
del
cereale
A
deve essere
non
superiore
a 1200
euro.
Analogamente,
per
il cereale
B
si ottiene
$7y_1
+ 2y_2
+
y_3
\leq
750.$
I prezzi
di vendita
dev
ono
essere
non
negativi
ed
il ricavato
della
vendita
è dato
$8y_1 + 15y_2 + 3y_3$
(dato
che
8, 15
e 3 sono
il minimo
numero
di pillole
di carboidrati,
proteine
e vitamine
necessari
alla
corretta alimentazione
del
pollo):
il problema
del
venditore
di pillole
per polli è dunque

\begin{equation}\begin{split}
\text{max}\qquad\qquad\ 8y_1+15y_2+3y_3\\
\text{} \qquad
\begin{split} 
&5y_1+4y_2+2y_3\leq 1200\\
&7y_1+2y_2+y_3\leq 750\\
&y_1,y_2,y_3\geq0.
\end{split}
\end{split}
\end{equation}

I due problemi sono riassunti nella seguente tabella

\begin{center}
	
\begin{tabular}{l*{4}{c}}
	
	& $x_1$ & $x_2$ & & max \\
	$y_1$ & 5 & 7 & & 8  \\
	$y_2$ & 4 & 2 & $\geq$ & 15 \\
	$y_3$ & 2 & 1 & & 3 \\
	& $\leq$ \\
	min & 1200 & 750 &&
	
\end{tabular}
\end{center}

La
coppia
di problemi
appena
costruita
gode
di un'importante proprietà:  comunque
si scelgano
i prezzi
$y_1$
,
$y_2$
e
$y_3$
delle
pillole,
il ricavo del
venditore
di pillole
è sicuramente
minore
o  uguale
al  costo
di
qualsiasi
dieta
ammissibile
che
l'allevatore
di polli
possa
ottenere
dai
due
mangimi.
Infatti,
i vincoli
del
venditore
di pillole
assicurano
che le pillole
siano
più convenienti dei
singoli
mangimi,
e quindi
anche
di ogni
loro
combinazione.
La
proprietà può essere
vericata
algebricamente nel
seguente modo:
moltiplicando
per
$x_1$
e
$x_2$
i due
vincoli
del
problema
del
venditore
di pillole
e sommando
le due
diseguaglianze
così ottenute,
si ha 
$$x_1(5y_1+4y_2+2y_3)+x_2(7y_1+2y_2+y_3) \leq 1200x_1+750x_2$$

Riordinando
i termini
in mo
do
da
mattere
in evidenza
le variabili
$y_i$, si ottiene

$$ y_1(5x_1+7x_2)+y_2(4x_1+2x_2)+y_3(2x_1+x_2) \leq 1200x_1+750x_2.$$
A questo
punto
possiamo
utilizzare
i vincoli
del
problema
della
dieta
per minorare
le
quantità tra
parentesi,
ottenendo

$$8y_1+15y_2+3y_3 \leq 1200x_1+750x_2.$$

Il costo
di qualsiasi
dieta
ammissibile
fornisce
quindi
una
valutazione
superiore
del
massimo
ricavo del
venditore
di pillole
e analogamente qualsiasi
ricavo ammissibile
per il venditore
di pillole
fornisce
una
valutazione
inferiore
del
costo
della
miglior
dieta
possibile.
Ciascuno
dei
due
problemi
fornisce
quindi,
attra
verso
il costo
di qualsiasi
soluzione
ammissibile,
una
valutazione
(inferiore
o superiore)
del
valore
ottimo
dell'altro
problema.
\\
\\
Come abbiamo visto, la formulazione del problema duale, ci permette di considerare lo
stesso problema dell’approvvigionamento delle sostanze necessarie all’allevatore dal punto
di vista alternativo (duale) di un produttore interessato ad agire come venditore (e non
come acquirente) nello stesso mercato.\\
\\
Spesso, i modelli di economia cadono nel regno della programmazione lineare. In particolare, molti teoremi riguardo l'equilibrio economico possono essere dedotti dal Teorema della Dualità e dal Teorema degli Scarti Complementari. 

\end{document}
